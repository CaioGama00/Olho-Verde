\documentclass{article}

% Language setting
% Replace `english' with e.g. `spanish' to change the document language
\usepackage[portuguese]{babel}

% Set page size and margins
% Replace `letterpaper' with `a4paper' for UK/EU standard size
\usepackage[letterpaper,top=2cm,bottom=2cm,left=3cm,right=3cm,marginparwidth=1.75cm]{geometry}

% Useful packages
\usepackage{amsmath}
\usepackage{float}
\usepackage{graphicx}
\usepackage{subcaption}
\usepackage[export]{adjustbox}
\usepackage[colorlinks=true, allcolors=blue]{hyperref}

\newcommand{\mycomment}[1]{}

\title{Entrega 3 - MAC0332}
\author{Nome: Renan Ryu Kajihara \\
        Nome: Caio Rodrigues Gama  \\
        Nome: Rachel Loriato Nazareth Franco  \\
        Nome: Felipe Kaneshiro de Souza  \\
        Nome: Tiago Gomes Dourado de Oliveira  \\
        Nome: Yesman Choque Mamani}

\begin{document}
\maketitle


\begin{abstract}

O presente relatório irá descrever as atividades realizadas no desenvolvimento do projeto "Olho Verde", uma plataforma web colaborativa para reporte e visualização de problemas urbanos na cidade de São Paulo. Entre os tópicos discutidos, estão: a arquitetura de software completa; a solução completa, implementada e testada; o plano de testes para as funcionalidades desenvolvidas; o relatório dos testes já realizados; o relatório de refatorações realizadas; a apresentação final com demonstração ao vivo da solução; a análise dos resultados; os desafios superados; e as futuras melhorias.

\end{abstract}

\newpage

\tableofcontents

\newpage

\section{Arquitetura de software completa}

lorem ipsum

\section{Solução completa}

\url{https://github.com/CaioGama00/Olho-Verde}

\mycomment{Seria bom adicionar mais algo aqui além do link para o repositório, como uma listagem das funcionalidades implementadas e talvez alguns diagramas de UML}

\section{Plano de testes para as funcionalidades desenvolvidas}

lorem ipsum

\section{Relatório dos testes já realizados}

lorem ipsum

\section{Relatório de refatorações realizadas}

Durante o desenvolvimento do projeto, uma das principais refatorações realizadas foi a reestruturação completa da camada de backend, que inicialmente estava implementada em um único arquivo monolítico (\texttt{server.js}) contendo toda a lógica da aplicação.

\subsection{Motivação}

O arquivo \texttt{server.js} original concentrava diversos componentes em um único local, incluindo definição de rotas HTTP, lógica de negócio, controladores, configurações do servidor e middlewares. Essa abordagem monolítica apresentava diversos problemas:

\begin{itemize}
    \item \textbf{Baixa manutenibilidade:} dificuldade em localizar e modificar funcionalidades específicas
    \item \textbf{Violação de princípios SOLID:} um único arquivo com múltiplas responsabilidades
    \item \textbf{Difícil testabilidade:} componentes fortemente acoplados
    \item \textbf{Escalabilidade comprometida:} crescimento desordenado do código
\end{itemize}

\subsection{Solução Implementada: Arquitetura MVC}

Para resolver esses problemas, foi implementada uma arquitetura baseada no padrão \textbf{MVC (Model-View-Controller)}, separando as responsabilidades em camadas bem definidas:

\vspace{0.3cm}

\begin{itemize}

\item \noindent\textbf{Controllers:} 

Camada responsável por receber as requisições HTTP, validar os dados de entrada, coordenar as chamadas aos services apropriados e retornar as respostas formatadas ao cliente. Os controllers atuam como intermediários entre as rotas e a lógica de negócio, mantendo o código de orquestração isolado e reutilizável.

\vspace{0.2cm}

\item \noindent\textbf{Services:} 

Camada que contém toda a lógica de negócio da aplicação. Os services são responsáveis por implementar as regras de negócio, processar dados, realizar validações complexas e orquestrar operações que envolvem múltiplas entidades. Esta camada é independente do protocolo HTTP, permitindo que a mesma lógica seja reutilizada em diferentes contextos.

\vspace{0.2cm}

\item \noindent\textbf{Routes:} 

Camada que define os endpoints da API e mapeia cada rota HTTP para o controller correspondente. Essa separação permite uma visão clara de todos os endpoints disponíveis na aplicação e facilita a documentação e manutenção das rotas.

\vspace{0.2cm}

\item \noindent\textbf{Models:} 

Camada que representa as entidades de dados e define a estrutura e o comportamento dos objetos de domínio. Os models são responsáveis pela interação com o banco de dados, definindo schemas, validações e métodos de acesso aos dados.

\end{itemize}

\subsection{Estrutura Resultante}

A nova estrutura de diretórios do backend ficou organizada da seguinte forma:

\vspace{0.3cm}

\begin{center}
\begin{minipage}{0.45\textwidth}
\begin{verbatim}
backend/
├── server.js              
├── controllers/          
│   ├── reportController.js
│   ├── userController.js
│   └── ...
├── services/             
│   ├── reportService.js
│   ├── userService.js
│   └── ...
├── routes/               
│   ├── reportRoutes.js
│   ├── userRoutes.js
│   └── ...
└── models/               
    └── ...
\end{verbatim}
\end{minipage}
\end{center}

\subsection{Benefícios Obtidos}

A refatoração trouxe melhorias significativas para o projeto:

\begin{itemize}
    \item \textbf{Modularização:} cada módulo possui uma responsabilidade clara e bem definida
    \item \textbf{Manutenibilidade:} facilidade em localizar e modificar funcionalidades específicas
    \item \textbf{Reusabilidade:} services podem ser compartilhados entre diferentes controllers
    \item \textbf{Testabilidade:} componentes isolados facilitam a criação de testes unitários
    \item \textbf{Colaboração:} múltiplos desenvolvedores podem trabalhar simultaneamente
    \item \textbf{Escalabilidade:} novas funcionalidades podem ser adicionadas sem comprometer a organização
\end{itemize}

\vspace{0.3cm}

\noindent Esta refatoração representa um marco importante na evolução do projeto, estabelecendo uma base sólida e profissional para o desenvolvimento futuro da plataforma Olho Verde.

\section{Apresentação final com demonstração ao vivo da solução}

lorem ipsum

\section{Análise final}

lorem ipsum

\subsection{Resultados}

lorem ipsum

\subsection{Desafios superados}

lorem ipsum

\subsection{Futuras melhorias}

lorem ipsum

\end{document}
