\documentclass{article}

% Language setting
% Replace `english' with e.g. `spanish' to change the document language
\usepackage[portuguese]{babel}

% Set page size and margins
% Replace `letterpaper' with `a4paper' for UK/EU standard size
\usepackage[letterpaper,top=2cm,bottom=2cm,left=3cm,right=3cm,marginparwidth=1.75cm]{geometry}

% Useful packages
\usepackage{amsmath}
\usepackage{float}
\usepackage{graphicx}
\usepackage{subcaption}
\usepackage[export]{adjustbox}
\usepackage[colorlinks=true, allcolors=blue]{hyperref}

\newcommand{\mycomment}[1]{}

\title{Entrega 3 - MAC0332}
\author{Nome: Renan Ryu Kajihara \\
        Nome: Caio Rodrigues Gama  \\
        Nome: Rachel Loriato Nazareth Franco  \\
        Nome: Felipe Kaneshiro de Souza  \\
        Nome: Tiago Gomes Dourado de Oliveira  \\
        Nome: Yesman Choque Mamani}

\begin{document}
\maketitle


\begin{abstract}

O presente relatório irá descrever as atividades realizadas no desenvolvimento do projeto "Olho Verde", uma plataforma web colaborativa para reporte e visualização de problemas urbanos na cidade de São Paulo. Entre os tópicos discutidos, estão: a arquitetura de software completa; a solução completa, implementada e testada; o plano de testes para as funcionalidades desenvolvidas; o relatório dos testes já realizados; o relatório de refatorações realizadas; a apresentação final com demonstração ao vivo da solução; a análise dos resultados; os desafios superados; e as futuras melhorias.

\end{abstract}

\newpage

\tableofcontents

\newpage

\section{Arquitetura de software completa}

lorem ipsum

\section{Solução completa}

\url{https://github.com/CaioGama00/Olho-Verde}

\mycomment{Seria bom adicionar mais algo aqui além do link para o repositório, como uma listagem das funcionalidades implementadas e talvez alguns diagramas de UML}

\section{Plano de testes para as funcionalidades desenvolvidas}

lorem ipsum

\section{Relatório dos testes já realizados}

lorem ipsum

\section{Relatório de refatorações realizadas}

lorem ipsum

\section{Apresentação final com demonstração ao vivo da solução}

lorem ipsum

\section{Análise final}

lorem ipsum

\subsection{Resultados}

lorem ipsum

\subsection{Desafios superados}

lorem ipsum

\subsection{Futuras melhorias}

lorem ipsum

\end{document}
